\documentclass[journal, a4paper]{IEEEtran}

% some very useful LaTeX packages include:

%\usepackage{cite}      % Written by Donald Arseneau
                        % V1.6 and later of IEEEtran pre-defines the format
                        % of the cite.sty package \cite{} output to follow
                        % that of IEEE. Loading the cite package will
                        % result in citation numbers being automatically
                        % sorted and properly "ranged". i.e.,
                        % [1], [9], [2], [7], [5], [6]
                        % (without using cite.sty)
                        % will become:
                        % [1], [2], [5]--[7], [9] (using cite.sty)
                        % cite.sty's \cite will automatically add leading
                        % space, if needed. Use cite.sty's noadjust option
                        % (cite.sty V3.8 and later) if you want to turn this
                        % off. cite.sty is already installed on most LaTeX
                        % systems. The latest version can be obtained at:
                        % http://www.ctan.org/tex-archive/macros/latex/contrib/supported/cite/

\usepackage{graphicx}   % Written by David Carlisle and Sebastian Rahtz
                        % Required if you want graphics, photos, etc.
                        % graphicx.sty is already installed on most LaTeX
                        % systems. The latest version and documentation can
                        % be obtained at:
                        % http://www.ctan.org/tex-archive/macros/latex/required/graphics/
                        % Another good source of documentation is "Using
                        % Imported Graphics in LaTeX2e" by Keith Reckdahl
                        % which can be found as esplatex.ps and epslatex.pdf
                        % at: http://www.ctan.org/tex-archive/info/

%\usepackage{psfrag}    % Written by Craig Barratt, Michael C. Grant,
                        % and David Carlisle
                        % This package allows you to substitute LaTeX
                        % commands for text in imported EPS graphic files.
                        % In this way, LaTeX symbols can be placed into
                        % graphics that have been generated by other
                        % applications. You must use latex->dvips->ps2pdf
                        % workflow (not direct pdf output from pdflatex) if
                        % you wish to use this capability because it works
                        % via some PostScript tricks. Alternatively, the
                        % graphics could be processed as separate files via
                        % psfrag and dvips, then converted to PDF for
                        % inclusion in the main file which uses pdflatex.
                        % Docs are in "The PSfrag System" by Michael C. Grant
                        % and David Carlisle. There is also some information
                        % about using psfrag in "Using Imported Graphics in
                        % LaTeX2e" by Keith Reckdahl which documents the
                        % graphicx package (see above). The psfrag package
                        % and documentation can be obtained at:
                        % http://www.ctan.org/tex-archive/macros/latex/contrib/supported/psfrag/

%\usepackage{subfigure} % Written by Steven Douglas Cochran
                        % This package makes it easy to put subfigures
                        % in your figures. i.e., "figure 1a and 1b"
                        % Docs are in "Using Imported Graphics in LaTeX2e"
                        % by Keith Reckdahl which also documents the graphicx
                        % package (see above). subfigure.sty is already
                        % installed on most LaTeX systems. The latest version
                        % and documentation can be obtained at:
                        % http://www.ctan.org/tex-archive/macros/latex/contrib/supported/subfigure/

\usepackage{url}        % Written by Donald Arseneau
                        % Provides better support for handling and breaking
                        % URLs. url.sty is already installed on most LaTeX
                        % systems. The latest version can be obtained at:
                        % http://www.ctan.org/tex-archive/macros/latex/contrib/other/misc/
                        % Read the url.sty source comments for usage information.

%\usepackage{stfloats}  % Written by Sigitas Tolusis
                        % Gives LaTeX2e the ability to do double column
                        % floats at the bottom of the page as well as the top.
                        % (e.g., "\begin{figure*}[!b]" is not normally
                        % possible in LaTeX2e). This is an invasive package
                        % which rewrites many portions of the LaTeX2e output
                        % routines. It may not work with other packages that
                        % modify the LaTeX2e output routine and/or with other
                        % versions of LaTeX. The latest version and
                        % documentation can be obtained at:
                        % http://www.ctan.org/tex-archive/macros/latex/contrib/supported/sttools/
                        % Documentation is contained in the stfloats.sty
                        % comments as well as in the presfull.pdf file.
                        % Do not use the stfloats baselinefloat ability as
                        % IEEE does not allow \baselineskip to stretch.
                        % Authors submitting work to the IEEE should note
                        % that IEEE rarely uses double column equations and
                        % that authors should try to avoid such use.
                        % Do not be tempted to use the cuted.sty or
                        % midfloat.sty package (by the same author) as IEEE
                        % does not format its papers in such ways.

\usepackage{amsmath}    % From the American Mathematical Society
                        % A popular package that provides many helpful commands
                        % for dealing with mathematics. Note that the AMSmath
                        % package sets \interdisplaylinepenalty to 10000 thus
                        % preventing page breaks from occurring within multiline
                        % equations. Use:
%\interdisplaylinepenalty=2500
                        % after loading amsmath to restore such page breaks
                        % as IEEEtran.cls normally does. amsmath.sty is already
                        % installed on most LaTeX systems. The latest version
                        % and documentation can be obtained at:
                        % http://www.ctan.org/tex-archive/macros/latex/required/amslatex/math/



% Other popular packages for formatting tables and equations include:

%\usepackage{array}
% Frank Mittelbach's and David Carlisle's array.sty which improves the
% LaTeX2e array and tabular environments to provide better appearances and
% additional user controls. array.sty is already installed on most systems.
% The latest version and documentation can be obtained at:
% http://www.ctan.org/tex-archive/macros/latex/required/tools/

% V1.6 of IEEEtran contains the IEEEeqnarray family of commands that can
% be used to generate multiline equations as well as matrices, tables, etc.

% Also of notable interest:
% Scott Pakin's eqparbox package for creating (automatically sized) equal
% width boxes. Available:
% http://www.ctan.org/tex-archive/macros/latex/contrib/supported/eqparbox/

% *** Do not adjust lengths that control margins, column widths, etc. ***
% *** Do not use packages that alter fonts (such as pslatex).         ***
% There should be no need to do such things with IEEEtran.cls V1.6 and later.


% Your document starts here!
\begin{document}

% Define document title and author
	\title{Encoding a Sudoku as a k-SAT problem}
	\author{Firstname Lastname}
	\markboth{Hauptseminar Digitale Kommunikationssysteme}{}
	\maketitle

% Write abstract here
\begin{abstract}
	This paper aims to examine the influence of the clauses length inside a Sudoku CNF encoding on the performance of a SAT solver.
\end{abstract}


\section{Introduction}
	


\section{Hypothesis}
	Our hypothesis is: \textit{"Encoding the Sudoku as a k-SAT problem, for $k _ 3, \dots$, will get better performances (in terms of CPU time, number of decisions and conflict) during the resolution with a SAT solver, compared to the naive encoding".} We were motived to investigate in this direction by the results of a paper dealing with general k-SAT problems and the search for hard problems.

\section{Experimental setup}
	
\subsubsection*{A. The naive encoding}
	
	The naive encoding was implementend by defining the functions \textit{at least one} and \textit{at most one}, which encode the concept of having one and only one value among the considered set. Then we applied this function to each single cell, to ensure only one value was assigned, to lines, to rows and finally to blocks. This approach returns clauses with length 2, that avoid the presence of more than one value in the cell, and clauses of length $n$, namely the dimension of the Sudoku. 
	
\subsubsection*{B. The k-SAT encoding}
	
	In order to riformulate the problem as a $k$-SAT problem, we introduced new dummy variables in the following way:
	
\begin{equation*}
	(a \lor b \lor c \lor d ) = (a \lor b \lor z) \land (c \lor d \lor \neg z).
\end{equation*}
	
	This increases the number of clauses and variables, but the length is reduced. We tried to state the problem as a k-SAT for $k \in \{ 3, 4\}$. 
	
\subsubsection*{C. The SAT solver}
	We choose to use the SAT solver miniSAT. It accepts files written in CNF format and returns the solution and a number of useful statistics. We were interested in the CPU time, the number of conflicts and the number of decisions.
	
\subsubsection*{D. The database}
	
	We used a database collected from a website. We were interested in analyzing the problem for Sudoku of different sizes, hence we scraped sudoku of size 9, 16 and 25.
	
\subsubsection*{D. The code}
	
	The code we wrote scrapes the sudoku from the website, then it encodes them in CNF format and reduces the encoding to a k-SAT problem. Finally miniSAT is used to solve the sudoku and gather statistics.
	

\section{Experimental results}
	
	For each sudoku size, we applied the procedure for k = 3,4 on 500 sudoku. Then we represented the statistics comparison in some histogram plots. The plot show that the performance is not increased with the new encoding. For what concerns the conflicts, there is almost no difference, but for the CPU time and the memory usage, it is clear that the SAT solver performance gets worse. Some further research confirmed that the length of the clauses is much less important than the number of clauses and variables.
	
	
Here we can see the comparison between 16x16 sudokus encoded in the naive way and as a 3-SAT and a 4-SAT.

\includegraphics[height=6cm]{16x16.png}

\includegraphics[height=5.7cm]{16x16_4SAT.png}



\section{Interpretation}




\section{Conclusion, summary, future work}
	




\begin{thebibliography}{5}

	%Each item starts with a \bibitem{reference} command and the details thereafter.
	\bibitem{k-SAT paper} % Transaction paper
		k-SAT paper

	\bibitem{miniSAT} % Conference paper
	
		miniSAT
	
	\bibitem{Sudoku database}
		Sudoku database


\end{thebibliography}

% Your document ends here!
\end{document}